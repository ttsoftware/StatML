\documentclass[12pt, a4paper]{article}
\usepackage{array}
\usepackage{longtable}
\usepackage[table]{xcolor}
\usepackage{hyperref}
\usepackage{float}
\usepackage[utf8]{inputenc}
\usepackage{graphicx}
\usepackage{amsmath}
\usepackage{amssymb}
\usepackage{amsfonts}
\usepackage[margin=1 in]{geometry}
\usepackage{color}
\usepackage{caption}
\usepackage{pdflscape}
\usepackage{fancyhdr}

\title{Assignment 3: Foundations\\Statistical Methods for Machine Learning}
\author{Troels Thomsen - qvw203\\Rasmus Haarslev - nkh877\\Allan Martin Nielsen - jcl187}

\setlength\parindent{0pt}		% noindent through whole document
\usepackage[parfill]{parskip}	% extra linebreak on new paragraph

\begin{document}
\pagestyle{empty}
\maketitle
\pagenumbering{gobble}
\newpage

\tableofcontents
\newpage

\pagenumbering{arabic}
\pagestyle{fancy}
\fancyhead[LO,LE]{qvw203 - nkh877 - jcl187}
\fancyhead[RO, RE]{Assignment 3}

\section*{II.1.1}
The derivative of the tansfer function can be found, by using the qoutient rule and noting that $\frac{d|a|}{da}=\frac{a}{|a|}|a\neq 0$:\\
\begin{eqnarray}
	h(x) &=& \frac{g(x)}{f(x)} | f(x) \neq 0\\
	h'(x) &=& \frac{g'(x)f(x)-g(x)f('x)}{(f(x))^2} | f(x) \neq 0
\end{eqnarray}
 
We can then find the derivative to the transfer function, by handling the numerator and denominator as functions and inserting them in (2):
\begin{eqnarray}
	g(a) &=& a\\
	f(a) &=& 1+|a|\\
	h'(a) &=& \frac{1\cdot (1+|a|)-a\frac{a}{|a|}}{(1+|a|)^2}\\
	h'(a) &=& \frac{1+|a|-\frac{a^2}{|a|}}{(1+|a|)^2}\\
	h'(a) &=& \frac{1+|a|-|a|}{(1+|a|)^2}\\
	h'(a) &=& \frac{1}{(1+|a|)^2}\\
\end{eqnarray}
The step from (6) to (7) we use the property of $\frac{a^2}{|a|} = |a|\ ,\ a\neq 0$


Note that (9) derivation only covers the cases where $a\neq 0$.  
The case where $a=0$, the function will yield 1, since it will have the form:

\begin{equation}
	h'(0) = \frac{1}{(1+|0|)^2} = \frac{1}{(1)^2} = 1
\end{equation}

Our neural network does not work. At some certain starting weights we can correctly predict our target value.\\
But for the most part, our randomized starting weights result in our prediction converging towards 1.

We can not figure out why this happens.
\section*{II.1.2}
\section*{II.2.1}

\[Training\ mean = \left(
\begin{array}{c}
155.96038775510206\\
204.821193877551\\
115.05862244897961\\
0.0059978571428571424\\
4.2887755102040829e-05\\
0.0032041836734693872\\
0.0033154081632653059\\
0.0096129591836734696\\
0.027739999999999997\\
0.26240816326530614\\
0.014676122448979591\\
0.016614489795918366\\
0.021988061224489795\\
0.044028163265306119\\
0.022639081632653057\\
22.000704081632652\\
0.49481960204081638\\
0.71568976530612249\\
-5.7637275306122442\\
0.21479572448979595\\
2.3657628673469384\\
0.19970881632653059
\end{array} 
\right) \]

\[Training\ variance = \left(
\begin{array}{c}
6.656095147075177\\
9.907169276818067\\
6.764287782442528\\
0.06287905927130613\\
0.0054852153402680905\\
0.04862574481849449\\
0.047705745079147835\\
0.08421604431724898\\
0.12609107329434685\\
0.4033549429662795\\
0.09301711006530504\\
0.10062890858480358\\
0.11529198649009464\\
0.16110723217228484\\
0.1725656260172071\\
2.015742161196478\\
0.31865601236010616\\
0.23626419183059846\\
1.0150922142264232\\
0.27531432075896684\\
0.6077867220135906\\
0.28566070025500484
\end{array} 
\right) \]


\[Normalized\ test\ mean = \left(
\begin{array}{c}
-0.078579309878629822\\
-0.15804161512095821\\
0.055623112213447443\\
0.11318387059617592\\
0.071573768538016072\\
0.086914890627931535\\
0.11567238855039749\\
0.087015530078705133\\
0.24898213522414622\\
0.24518734019950522\\
0.22956619607003162\\
0.2508905077913246\\
0.31660825718863211\\
0.22960283397634887\\
0.14905702097923168\\
-0.056763460231311701\\
0.073567655778488658\\
0.086766982724276354\\
0.15477244610125082\\
0.31069454970868138\\
0.087416429507011437\\
0.16857660314195763
\end{array}
\right) \]

\[Normalized\ test\ variance = \left(
\begin{array}{c}
0.9250287228732689\\
0.9195246016888662\\
0.945028639553341\\
1.1877778096537361\\
1.1361125965666579\\
1.209031894686311\\
1.1774789752769503\\
1.2091643546449475\\
1.1537541194986605\\
1.1629226629758114\\
1.1447548293853007\\
1.154710445495337\\
1.2165286991929372\\
1.1447780591529846\\
1.277443113007099\\
1.0800818556277136\\
1.0200472935835834\\
0.9875983134091282\\
1.0502167122382409\\
1.0804588903995263\\
1.0318272630890424\\
1.0906017110242643
\end{array}
\right) \]

\section*{II.2.2}
For this assignment, we chose to use the SKLearn-module for Python, which is a Python interface for the LIBSVM library, that can implement Support Vector Machines.

For cross-validation we used the same procedure as in the NN-Classifier from assignment 1 and modified our implementation to fit this classifier.

Initially the algorithm creates $S$ pairs of test and training sets .

Then it iterates through each value of $\gamma$ and for each of these values it iterates over the values of $C$ and calculates the loss for each test/training pair, by giving the data-points from the test set, to the SVM-fitter, that is trained with the corresponding training set.

If the loss is less than the previous best $\gamma/C$-pair, then this will be the new best $\gamma/C$-pair

The program then returns this $\gamma/C$-pair, and the corresponding loss.

The $\gamma$ and $C$, that we found for the raw, and normalized data is as follows:

\textit{Raw data:}
\begin{itemize}
	\item $\gamma_{best} = 0.0001$
	\item $C_{best} = 1$
\end{itemize}

\textit{Normalized data:}
\begin{itemize}
	\item $\gamma_{best} = 0.01$
	\item $C_{best} = 1$
\end{itemize}

With these hyperparameters, we get the following losses:

\textit{Raw data:}
\begin{itemize}
	\item Raw training set: 0.15306122449
	\item Raw test set: 0.20618556701
\end{itemize}

\textit{Normalized data:}
\begin{itemize}
	\item Normalized training set: 0.122448979592
	\item Normalized test set: 0.134020618557
\end{itemize}

As can be seen, we achieve a much lower loss after normalization.

\section*{II.2.3}

\subsection{II.2.3.1}

\textit{Raw data:}
\begin{itemize}
	\item Bounded Support Vectors = 32
	\item Free Support Vectors = 15
\end{itemize}

\textit{Normalized data:}
\begin{itemize}
	\item Bounded Support Vectors = 44
	\item Free Support Vectors = 6
\end{itemize}

The C parameter is a regularization variable, which indicates how much the SVM should avoid misclassifying the training examples. If C is large, the SVM will try to choose hyperplanes with a smaller margin, if that hyperplane does a better job at classifying. On the other hand, a small C will look for a hyperplane with a larger margin, even though it may misclassify more of the data points. Extremely small values of C will often lead to misclassification even if the training data is linearly separable.

\subsection{II.2.3.2}

When the amount of training examples increases the amount of support vectors will grow asymptotically linearly. This development is due to the added data will propagate towards, but never reach a still (percentage of training  examples), even though there is noise on the data.

On large applications this will mean the linear growth diminishes with the amount of dimensions in the data. It will however always grow.

\end{document}

